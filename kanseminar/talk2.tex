%:
\documentclass[11pt, oneside]{article}   	% use "amsart" instead of "article" for AMSLaTeX format
\usepackage{geometry}                		% See geometry.pdf to learn the layout options. There are lots.
\geometry{letterpaper}                   		% ... or a4paper or a5paper or ... 
%\geometry{landscape}                		% Activate for rotated page geometry
%\usepackage[parfill]{parskip}    		% Activate to begin paragraphs with an empty line rather than an indent
\usepackage{graphicx}				% Use pdf, png, jpg, or eps§ with pdflatex; use eps in DVI mode
								% TeX will automatically convert eps --> pdf in pdflatex		
\usepackage{amssymb}
\usepackage{diagbox}
\usepackage{amsmath}
\usepackage{amsthm}
\usepackage{enumerate}
\theoremstyle{definition}
\newtheorem*{defn}{Definition}
\newtheorem*{prop}{Proposition}
\newtheorem*{eg}{Example}
\newtheorem{thm}{Theorem}
\newtheorem*{corol}{Corollary}
\newtheorem{ex}{Exercise}[section]
{\theoremstyle{plain}
\newtheorem*{rmk}{Remark}
\newtheorem*{rmks}{Remarks}
\newtheorem*{lt}{Last time}
}
\newtheorem*{lem}{Lemma}
\usepackage{color}
\usepackage{CJK}
\title{Nilpotence and Stable Homotopy Theory II}
\author{Xiyu Zhai}
\date{}							% Activate to display a given date or no date

\begin{document}
\maketitle
\tableofcontents

This note is adapted from Gabriel Angelini-Knoll's note

\section{Notions}

\subsection{Smash Product}



\begin{defn}[Smash Product]
Let $X$ and $Y$ be spaces with base points $x_0,y_0$ resp. The smash product $X\wedge Y$ is the quotient of $X\times Y$ obtained by collapsing $X\times \{y_0\}\cup \{x_0\}\times Y$ to a single point.

$X^{(k)}$ is the $k$-fold iterated smash product of $X$ with itself.

For $f:X\to Y$, $f^{(k)}$ denote the evident map from $X^{(k)}$ to $Y^{(k)}$
	
\end{defn}

\subsection{Suspension}

\begin{defn}[Suspension]
	For $X$ a space with basepoint, $\Sigma X:=X\wedge S^1$

	For $f: X\to Y$, define $\Sigma f:= f\wedge 1_{S^1}: \Sigma X\to \Sigma Y$.

	These constructions can be iterated, denoted by $\Sigma^i X, \Sigma^i f$.

	If $\Sigma^i f$ for some $i$ we say $f$ is \textbf{stably null homotopic}; otherwise it is \textbf{stably essential}.

	One can use the suspension to convert $[X,Y]$ to a graded object $[X,Y]_*$, where $[X,Y]_i=[\Sigma^i X,Y]$. It's also useful to consider the group of stable homotopy classes of maps, $[X,Y]^S_i = \lim\limits_{\rightarrow}[\Sigma^{i+j}X,\Sigma^jY]$.
\end{defn}

\subsection{Spectrum}


\begin{defn}[Spectrum]
	A spectrum is a sequence of spaces weakly equivalent to CW-complexes $\{E_i\}_{i\in \mathbb{Z}}$ with structure maps $\Sigma E_n\to E_{n+1}$
\end{defn}

\begin{rmk}
	$\Sigma E_n\to E_{n+1}$ corresponds to $E_n\to \Omega E_{n+1}$. When this is a weak equivalence, this is an $\Omega$-spectrum.
\end{rmk}

\begin{defn}[Naive Smash Product for spectra]
	For spectra $E$ and $F$, the naive smash product is defined by
	\begin{equation}
		(E\wedge F)_{2n}=E_n\wedge F_n
	\end{equation}
	\begin{equation}
		(E\wedge F)_{2n+1}= \Sigma E_n\wedge F_n
	\end{equation}
	where the map
	\begin{equation}
		\Sigma E_n\wedge \Sigma F_n = \Sigma(E\wedge F)_{2n+1}\to (E\wedge F)_{2n+2}=E_{n+1}\wedge F_{n+1}
	\end{equation}
	is the smash product of the maps $\Sigma E_n\to E_{n+1}$ and $\Sigma F_n\to F_{n+1}$.
\end{defn}

\begin{rmk}
	The correct definition is rather difficult, see J. F. Adams. Stable Homotopy and Generalised Homology. University of Chicago Press, Chicago, 1974.

	But here the only smash products we need are with finite spectra,
which are always suspension spectra, so the naive definition is
adequate.
\end{rmk}

\begin{defn}[Ring Spectrum]
	A ring spectrum is a spectrum $E$ together with a multiplication map
	\begin{equation}
		\mu: E\wedge E\to E
	\end{equation}

	and a unit map
	\begin{equation}
		\eta:S\to E,
	\end{equation}

	These maps have to satisfy associativity and unitality conditions up to homotopy,
	\begin{equation}
		\mu\circ (id\wedge \mu)\sim \mu\circ(\mu\wedge id)
	\end{equation}

	and

	\begin{equation}
		\mu\circ(id\wedge \eta)\sim id\sim \mu(\eta\wedge id).
	\end{equation}

	Examples of ring spectra include those from singular homology with coefficients in a ring, complex cobordism, K-theory and Morava K-theory.
\end{defn}

\begin{defn}[Function]
	A function from spectrum $E$ to spectrum $F$ is a sequence of maps from $E_n$ to $F_n$ commuting with structure maps.
\end{defn}

\begin{defn}[Homotopy]
	 A homotopy of maps between spectra corresponds to a map $(E\wedge I^+)\to F$ where $I^+$ is the disjoint union $[0,1]\sqcup \{*\}$ with $*$ taken to be the basepoint.
\end{defn}

\begin{defn}[Stable Homotopy Category]
	 the category whose objects are spectra and whose morphisms are homotopy classes of maps between spectra
\end{defn}

\begin{defn}[Brown representability]
	This is the true clincher about spectra. For $E$ a spectrum and $X$ a space, we define
	\begin{equation}
		E_*X=[S,E\wedge X]_*
	\end{equation}
	and
	\begin{equation}
		E^*=[X,E]_{-*}
	\end{equation}

	For a pair $(A,X)$, we can just define $E_*(A,X)=E_*(X\cup CA)$.

	The functors $E_*$ ($E^*$) are homology(cohomology) theories on the category of spaces.
\end{defn}

\section{Nilpotence Theorems}

\subsection{Definition of }

\begin{defn}[Nilpotence]
	Various forms of nilpotence definitions.

	\begin{enumerate}[1]
		\item A map $f:S^n\to R$ with $R$ a ring spectrum is nilpotent if $f\in \pi_*R$ is nilpotent.
		\item A map $f: \Sigma^jF\to F$ is \textit{nilpotent} if $f^t:\Sigma^{jt}F\to F$ is null for some $t\gg 0$.
		\item A map $f: F\to X$ is \textit{smash nilpotent} if $f^{(n)}:F^{(n)}\to X{(n)}$ is null for some $n\gg 0$.
	\end{enumerate}
\end{defn}

\subsection{First cohomology theory detecting nilpotence}

We would like to have some cohomology theory that detects these kind of maps. Ravenel conjectured that $MU$ is the right cohomology to do the job. In 1988 Devinatz, Hopkins and Smith succeeded in proving that this is the case:

\begin{thm}
	[DHS, Nilpotence and Stable Homotopy Theory I]
\label{DHS}
	Suppose that $F$ is finite. Then for any of the three notions, $f$ is nilpotent if $MU_*(f)=0$.
\end{thm}

\begin{rmk}
	If the range of $f$ is $p$-local, then $f$ is nilpotent if $BP_*(f)=0$.
\end{rmk}
 
	We can recover Nishida's theorem as a special case.

	Elements of positive degree of the homotopy groups of spheres are nilpotent.

	In fact, for any $f\in \pi_n(\mathbb{S})$ with $n>0$, its image in $MU_*(\mathbb{S})$ is p-torsion because of Serre finiteness theorem. But $MU_*(\mathbb{S})$ is the Lazard ring, a polynomial ring, so $MU_*(f)$ is zero. Then $f$ is nilpotent.
 

\subsection{Refined Nilpotence Theorem}

We can use the Morava K-theories $K(n)$ to break this computation into computations at each prime and each height.

\begin{defn}[Spectra related to BP] The study of a ring is often simplified by passage to its quotients and localizations. The same is true of ring spectra, though constructing quotients and localizations can be difficult. Luckily Brown-Peterson spectrum is easy, using Baas-Sullivan theory of bordism with singularities.

Recall that $BP_*\approx \mathbb{Z}_{(p)}[v_1,\cdots,v_n,\cdots]$ with $|v_n|=2p^{n}-2$ with $\{v_n\}$ the Hazewinkel generators. For $0<n<\infty$ the ring spectra $K(n)$ and $P(n)$ are defined by the isomorphisms
\begin{equation}
	K(n)_*\approx \mathbb{F}_p[v_n,v_n^{-1}],
\end{equation}

\begin{equation}
	P(n)_*\approx \mathbb{F}_p[v_n, v_{n+1},\cdots],
\end{equation}

we also set
\begin{equation}
	K(0)=H \mathbb{Q},
\end{equation}
and
\begin{equation}
	K(\infty)= H \mathbb{F}_p
\end{equation}

\begin{equation}
	P(0)_*\approx BP_*
\end{equation}
\end{defn}

\begin{thm}
	[Nilpotence theorem refined]

	We will now restrict to $p$-local spectra, though we could have used any finite spectrum and then checked this at its $p$-localizations for each $p$.

	\begin{enumerate}[(1)]
		\item A map $f:S^n\to R$ regarded as $f\in \pi_*(R)$ where $R$ is a ring spectrum is nilpotent iff $K(n)_*(f)$ is nilpotent for all $0\leq n\leq \infty$.
		\item A self map $f:\Sigma^kF\to F$ is nilpotent iff $K(n)_*(f)$ is nilpotent for all $0\le n<\infty$.
		\item A map $f:F\to X$ is smash nilpotent iff $K(n)_*(f)=0$ for all $0\le n<\infty$
	\end{enumerate}
\end{thm}

\section{Proof}

\subsection{(iii) $\Rightarrow$ (i)}

\begin{itemize}
	\item Suppose that $f: S^n\to R$ is nilpotent, then by definition $f\in \pi_*R$ is nilpotent. $K(n)_*(f)$ is the image of $f$ under the Hurewicz homomorphism from $\pi_* R$ to $K(n)_*R$, then $K(n)_*(f)$ is nilpotent.
	\item Suppose that $K(n)_*(f)$ is nilpotent for all $0\le n\le\infty$. Then $f$ is smash nilpotent. Note that $f^m$ is obtained from composition of the smash product $f^{(m)}: (S^n)^{(m)}\to R^{(m)}$ with the ring multiplication $\mu: R\times R\to R$, so $f^m$ is null homotopic when $f^{(m)}$ is null homotopic then $f$ is nilpotent.
\end{itemize}

\subsection{(i) $\Rightarrow$ (ii)}

Here we introduce Spanier-Whitehead Duality.

For $f: \Sigma^i F\to F$, we have a dual map
\begin{equation}
	D(f):S^i\to D(F)\wedge F =: R
\end{equation}

where $D(F)=Map(F,S)$ the mapping spectrum from $F$ to $S$.

$R$ is a ring spectrum if $F$ is finite.

Then part ii) follows from part i) since multiplication in the rings
\begin{equation}
	\pi_*X\wedge DX \text{ and }K(n)_*X\wedge DX
\end{equation}
corresponds, under Spanier-Whitehead duality, to composition in
\begin{equation}
	[X,X]_*\text{ and }End_{K(n)*}(K(n)_*X)
\end{equation}

\subsection{(iii)}

Replacing
\begin{equation}
	f:F\to X
\end{equation}
with
\begin{equation}
	Df:S^0\to DF\wedge X
\end{equation}
in part iii) if necessary, we may assue that $F=S^0$. 

\begin{itemize}
	\item Suppose that $f$ is smash nilpotent, then $1_{BP}\wedge f^{(m)}$ is null homotopic for $m\gg 0$. Then by Bousfield equivalence
	\begin{equation}
		\langle BP\rangle=\langle K(0)\rangle\vee\cdots\vee \langle K(n)\rangle\vee \langle 
		P(n+1)\rangle,
	\end{equation}
	we have
	\begin{equation}
		K(n)_*(f^{(m)})=0.
	\end{equation}

	Since $K(n)_*$
\end{itemize}

\begin{defn}[Bousfield class]
	A spectrum $X$ is in the Bousfield class of $Y$, denoted by $\langle Y\rangle$ if $X\wedge Y$ is nontrivial.
\end{defn}
 
 	The key is to use
	\begin{equation}
		\langle BP\rangle= \langle K(0) \vee\cdots\vee K(n) \vee P(n)\rangle
	\end{equation}

	Also $\lim\limits_{\rightarrow}P(n)=HF_p=K(\infty)$.
 


\end{document}