%:
\documentclass[11pt, oneside]{article}   	% use "amsart" instead of "article" for AMSLaTeX format
\usepackage{geometry}                		% See geometry.pdf to learn the layout options. There are lots.
\geometry{letterpaper}                   		% ... or a4paper or a5paper or ... 
%\geometry{landscape}                		% Activate for rotated page geometry
%\usepackage[parfill]{parskip}    		% Activate to begin paragraphs with an empty line rather than an indent
\usepackage{graphicx}				% Use pdf, png, jpg, or eps§ with pdflatex; use eps in DVI mode
								% TeX will automatically convert eps --> pdf in pdflatex		
\usepackage{amssymb}
\usepackage{diagbox}
\usepackage{amsmath}
\usepackage{amsthm}
\usepackage{bbm}
\usepackage{enumerate}
\theoremstyle{definition}
\newtheorem*{defn}{Definition}
\newtheorem*{prop}{Proposition}
\newtheorem*{eg}{Example}
\newtheorem*{thm}{Theorem}
\newtheorem*{corol}{Corollary}
\newtheorem{ex}{Exercise}[section]
{\theoremstyle{plain}
\newtheorem*{rmk}{Remark}
\newtheorem*{rmks}{Remarks}
\newtheorem*{lt}{Last time}
}
\newtheorem{lem}{Lemma}
\usepackage{color}
\usepackage{CJK}
\title{Kan Seminar Talk1: Brown Representability Theorem about Cohomology Theories}
\author{Xiyu Zhai}
\date{}							% Activate to display a given date or no date

\begin{document}
\maketitle
\tableofcontents

\section{Introduction}

\subsection{Notions}

\begin{defn}[Reduced Suspension]
	$\Sigma X:=SX/\sim, (x_0,t)\sim (x_0,1)$.
\end{defn}

\begin{defn}[Mapping Cylinder]
	$M_f:=[0,1]\times X\sqcup Y/\sim, (x,1)\sim f(x), (x_0,t)\sim y_0$.
\end{defn}

\begin{rmk}
	$M_f$ (strongly) deformation retracts to $Y$ and $X$ is a subspace of $M_f$.
\end{rmk}

\begin{defn}[Reduced Mapping Cylinder]
	You can guess.
\end{defn}

\subsection{Motivations}

First motivation is the following fundamental relationship between singular cohomology and Eilenberg-MacLane spaces:

\begin{thm}
	[Theorem 4.57 the Homotopy Construction of Cohomology] There are natural bijections $T:\langle X, K(G,n)\rangle \to H^n(X;G)$ for all CW complexes $X$ and all $n>0$, with $G$ any abelian group. Such a $T$ has the form $T([f])=f^*(\alpha)$ for a certain distinguished class $\alpha\in H^n(K(G,n);G)$.
\end{thm}

\paragraph{Sketch of Proof} $K(G,n)$ is very concrete, use them to argue that $T\langle -, K(G,n)\rangle$ must agree with $H^n(K;G)$.

Question: in general, what condition we can put on a sequence of CW complexes $K_n$ such that $\langle X, K_n\rangle$ behaves like cohomology?

The natural isomorphism $h^n(X)\approx h^{n+1}(\Sigma X)$ leads to
\begin{equation}
	\langle X, K_n\rangle \approx \langle \Sigma X, K_{n+1}\rangle\approx \langle X, \Omega K_{n+1}\rangle
\end{equation}

\begin{defn}[$\Omega$-spectrum]
	A sequence of CW complexes $K_n$ such that
	$K_n$ is weak homotopic to $\Omega K_{n+1}$.
\end{defn}

\begin{thm}
[Theorem 4.58 in Allen Hatcher] If $\{K_n\}$ is an $\Omega$-spectrum, then the functors $X\mapsto h^n(X)=\langle X, K_n\rangle$, $n\in \mathbb{Z}$, define a reduced cohomology theory on the category of basepointed CW complexes and basepoint-preserving maps.
\end{thm}

\begin{eg}
	[Bott periodicty-topological version]
	\begin{equation}
		BU\times \mathbb{Z}\sim \Omega^2 BU.
	\end{equation}
\end{eg}

\subsection{Axioms}

For a contravariant functor $h$ from $C$ (category of CW complexes with basepoint) to abelian groups, we define the following axioms

Axiom 1(homotopy invariance)

Axiom 2 (exact sequence)
for $A\hookrightarrow X$, exact sequence
		\begin{equation}
			h(X/A)\to h(X)\to h(A)\text{ is exact}
		\end{equation}

Axiom 3 (wedge) $h(\vee_\alpha X_\alpha)\simeq \prod_\alpha h(X_\alpha)$ induced by $i_\alpha$

\begin{rmk}
	These axioms imply Mayer-Vietoris, just like page 162 in Allen Hatcher's book.

	Suppose $X=A\cup B$, $a\in h(A), b\in h(B)$ and $a|_{A\cap B}=b|_{A\cap B}$ then there exists $x\in h(X)$ such that $a=x|_A,b=x|_B$.
\end{rmk}

\subsection{Main Results}

\begin{defn}[A Reduced Cohomology Theory on the category of CW complexes with basepoint]
	A sequence of functors $h^n, n\in \mathbb{Z}$ from $\mathcal{C}$ to abelian groups, together with natural isomorphisms $h^n(X)\approx h^{n+1}(\Sigma X)$ for all $X$ in $\mathcal{C}$ such that axiom (i) (ii) (iii) all hold for each $h^n$.
\end{defn}

\begin{thm}[Theorem 1]
	Every reduced cohomology theory on the category of basepointed CW complexes and basepoint-preserving maps has the form $h^n(X) = \langle X, K_n\rangle$ for some $\Omega$-spectrum $\{K_n\}$.
\end{thm}

\begin{thm}[Theorem 2]
	If $h$ is a contravariant functor from the category of connected base-pointed CW complexes to the category of pointed sets, satisfying the homotopy axiom $(i)$, the Mayer-Vietoris axiom, and the wedge axiom $(iii)$, then there exists a connected CW complex $K$ and an element $u\in h(K)$ such that the transformation $T_u: \langle X, K \rangle \to h(X), T_u(f)=f^*(u)$, is a bijection for all $X$.
\end{thm}

\section{Proof}

\subsection{Lemmas}

\begin{defn}[$\pi_*$-universal]
	For a pair $(K, u)$ with $K$ a basepointed connected CW complex and $u\in h(K)$ where $h$ satisfies the axioms of theorem 2, we call it $\pi_*$-universal if for any $i$, $T_u: \pi_i(K)\to h(S^i)$ is an isomorphism.
\end{defn}

\begin{lem}
	Given any pair $(Z, z)$ with $Z$ a connected CW complex and $z\in h(Z)$ there exists a $\pi_*$-universal pair $(K, u)$ with $Z$ a subcomplex of $K$ and $u|_Z = z$.
\end{lem}

\paragraph{Sketch of Proof}

Make $K$ by attaching cells to create and to kill.

\begin{lem}
	Let $(K, u)$ be a $\pi_*$-universal pair and let $(X, A)$ be a basepointed $CW$ pair. Then for each $x\in h(X)$ and each map $f: A\to K$ with $f^*(u)=x|_A$ there exists a map $g:X\to K$ extending $f$ with $g^*(u)=x$.
\end{lem}

We shall proceed in four steps:
\begin{enumerate}[(1)]
	\item prove that Lemma 1 $\land$ Lemma 2 $\Rightarrow$ Theorem 2
	\item prove that Theorem 2 $\Rightarrow$ Theorem 1
	\item prove Lemma 2 $\Rightarrow$ Lemma 1
	\item prove Lemma 1
\end{enumerate}

\subsection{Prove that Lemma 1 $\land$ Lemma 2 $\Rightarrow$ Theorem 2}

There are two things to prove, surjectiveness and injectiveness.

\paragraph {Use lemma 1 to get a $\pi_*$-universal pair $(K,u)$.}

\paragraph {Apply lemma 2 to $(X, pt)$ to get surjectiveness.}

\paragraph {Apply lemma 2 to $(X\times I, X\times \partial I)$ to get injectiveness.}

Suppose that $T_u(f_0)=T_u(f_1)$, that is $f^*_0(u)=f^*_1(u)$. Combine $f_0$ and $f_1$ to form a map from $X\times \partial I\to K$ and taking $x$ to be $p^*f_0^*(u)=p^*f_1^*(u)$ where $p$ is the projection $X\times I\to X$. Here $X\times I$ should be the reduced product, with basepoint $\times I$ collapsed to a point. Then the lemma gives a homotopy from $f_0$ to $f_1$.

\subsection{Prove that Theorem 2 $\Rightarrow$ Theorem 1}

\paragraph{Restrict to connected CW complexes.} This is okay because suspension is an isomorphism in any reduced cohomology theory, and the suspension of any CW complex is connected.

\paragraph{Obtain weak homotopy equivalence.} Since $h$ is a cohomology theory, we have naturally

\begin{equation}
	h^n(X)\approx h^{n+1}(\Sigma X),
\end{equation}

then naturally
\begin{equation}
	\langle X, K_n\rangle\approx\langle \Sigma X, K_{n+1}\rangle\approx\langle X, \Omega K_{n+1}\rangle
\end{equation}

Apply Yoneda's lemma to the natural bijection $\Phi: \langle X, K_n\rangle\approx\langle X, \Omega K_{n+1}\rangle$ we obtain $\epsilon_n:K_n\to \Omega K_{n+1}$ such that $\epsilon_n$
 induces $\Phi$. Then it follows easily that $\epsilon_n$ is a weak homotopy equivalence (by taking $X$ to be $S_n$).

(Note: Allen Hatcher doesn't use Yoneda here. He basically reproves Yoneda. Maybe he wants to be clearer.)

Now we get an $\Omega$-spectrum.

\paragraph{Verify $h^n(X)\approx \langle X, K_n\rangle$ is a group isomorphism.} Here $\langle X, K_n\rangle$ has the group structure that comes from identifying it with $\langle X, \Omega K_{n+1}\rangle\approx \langle \Sigma X, \Omega K_n\rangle$.

We have
\begin{equation}
	\psi:\Sigma X\to\Sigma X\vee \Sigma X
\end{equation}
by collapsing the middle.

This map induces the group structure on the right obviously.

But this also induces the group structure for $h$.

Consider $p_1, p_2$ that collapse one of $\Sigma X$ in $\Sigma X\vee \Sigma X$ resp.

The key trick is: $p_1\circ \psi\sim p_2\circ \psi\sim \mathbbm{1}$.

Then just apply Mayer-Vietoris to see that

(here all $+$ means the original group structure)

$\psi^*((a,b))=\psi^*((a,0))+\psi^*((0,b)) = \psi^* p_1^* a + \psi^* p_2^*b=a+b$.

In Allen Hatcher's book, this was already proved in some form in a previous section.

\subsection{Prove Lemma 1 $\Rightarrow$ Lemma 2}

Let $(K, u)$ be a $\pi_*$-universal pair and let $(X, A)$ be a basepointed $CW$ pair.
We want to prove for each $x\in h(X)$ and each map $f: A\to K$ with $f^*(u)=x|_A$ there exists a map $g:X\to K$ extending $f$ with $g^*(u)=x$.

\paragraph {Reduce to the case $f$ is the inclusion of a subcomplex.} This is okay because we can replace $K$ by the reduced mapping cylinder of $f$.

\paragraph {Sketch of Proof} Put $K$ and $X$ together to a common larger space $K'$ using Lemma 1 with some $u'\in h(K')$ such that $u'|_K=u$ and $u'|_X=x$. Then prove that $X\hookrightarrow K'$ deformation retracts to $X \to K\subset K'$.

\paragraph {Apply Lemma 1 to get $(K', u')$ from $Z=X\cup K$ with the two copies of $A$ identified.} By Mayer-Vietoris axiom, there exists $z\in h(Z)$ with $z|_X=x$ and $z|_K=u$. Introduce a $\pi_*$-universal pair $(K', u')$ extending from $(Z, z)$ by applying Lemma 1.

\paragraph {Show $X\hookrightarrow K'$ is homotopic to $g: X\to K$ rel $A$.} The inclusion $(K, u)\hookrightarrow (K', u')$ induces an isomorphism on homotopy groups since both $u$ and $u'$ are $\pi_*$-universal, so $K'$ deformation retracts onto $K$ by whitehead. Examining the long exact sequence for the homotopy of the triple $(A, K, K')$ it is not hard to see that the conditions for a relative version of whitehead theorem is satisfied, so $K'$ deformation retracts onto $K$ rel $A$ (this part is not mentioned in Hatchers' book, he just directly reaches this), then $X\hookrightarrow K'$ is homotopic to $g: X\to K\subset K'$ rel $A$.

Then $g^*(u)=x$ holds because
\begin{equation}
	g^*(u)=g^*(u'|_K)=u'|_X=(u'|_Z)|_K=z|_K=u.
\end{equation}

\subsection{Prove Lemma 1}

We first propose a weaker notion than $\pi_*$-universal.

\begin{defn}[$n$-universal]
	For a pair $(K, u)$ with $K$ a basepointed connected CW complex and $u\in h(K)$ where $h$ satisfies the axioms of theorem 2, we call it $n$-universal if for any $i<n$, $T_u: \pi_i(K)\to h(S^i)$ is an isomorphism and for $i=n$ is surjective.
\end{defn}

\begin{rmk}
	$\pi_*$-universal is $n$-universal for all $n$.
\end{rmk}

We construct $K$ from $Z$ by an inductive process of attaching cells.

Let $K_1=Z\vee\bigvee_\alpha S^1_\alpha$ where $\alpha$ ranges over the elements of $h(S^1)$.

\paragraph{Prove $(K_1, u_1)$ is $1$-universal.} Just by wedge axiom.

For the inductive step, suppose we already have $(K_n, u_n)$ with $u_n$ $n$-universal, $Z\subset K_n$ and $u_n|Z=z$.

Represent each element $\alpha$ in the kernel of $T_{u_n}:\pi_n(K_n)\to h(S^n)$ by a map $f_\alpha: S^n\to K_n$. Let $f=\vee_ \alpha f_\alpha:\vee_\alpha S^n_\alpha\to K_n$. Consider $C_f=M_f/\vee_\alpha S^n_\alpha$. Then set $K_{n+1}=C_f\vee_\beta S_\beta^{n+1}$ where $\beta$ ranges over $h(S^n+1)$.

We now choose $u_{n+1}\in h(K_{n+1})$ such that $u_{n+1}|_{K_n}=u_n$. This is possible by
\begin{enumerate}[(i)]
	\item first extend $u_n$ to $C_f$ using Mayer-Vietoris. $C_f$ can be cut in half, with one deforms to $K_n$ and another to a point and the intersection is $\vee_\alpha S_\alpha^n$.
	\item then extend $u_n$ over $K_{n+1}$ by wedge axiom.
\end{enumerate}

\paragraph{Prove $(K_{n+1}, u_{n+1})$ is $n+1$-universal.}

We have two goals:

\begin{enumerate}[(i)]
	\item For $i\le n$, $T_{u_{n+1}}:\pi_i(K)\to h(S^i)$ is an isomorphism.
	\item For $i=n+1$ $T_{u_{n+1}}:\pi_{i}(K)\to h(S^i)$ is surjective.
\end{enumerate}

Note that $K_{n+1}$ is obtained from attaching $n+1$-cells, $i<n$ is obvious because nothing changes.

For $i=n$, leave as an exercise. It's like a very basic manuvre.

For $i=n+1$ also quite obvious.
\appendix
\end{document}