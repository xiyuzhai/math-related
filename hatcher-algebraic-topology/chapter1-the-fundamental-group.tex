%:
\documentclass[11pt, oneside]{article}   	% use "amsart" instead of "article" for AMSLaTeX format
\usepackage{geometry}                		% See geometry.pdf to learn the layout options. There are lots.
\geometry{letterpaper}                   		% ... or a4paper or a5paper or ... 
%\geometry{landscape}                		% Activate for rotated page geometry
%\usepackage[parfill]{parskip}    		% Activate to begin paragraphs with an empty line rather than an indent
\usepackage{graphicx}				% Use pdf, png, jpg, or eps§ with pdflatex; use eps in DVI mode
								% TeX will automatically convert eps --> pdf in pdflatex		
\usepackage{amssymb}
\usepackage{diagbox}
\usepackage{amsmath}
\usepackage{amsthm}
\usepackage{enumerate}
\usepackage{bm}
\theoremstyle{definition}
\newtheorem*{defn}{Definition}
\newtheorem*{prop}{Proposition}
\newtheorem*{eg}{Example}
\newtheorem*{thm}{Theorem}
\newtheorem*{corol}{Corollary}
\newtheorem{ex}{Exercise}
{\theoremstyle{plain}
\newtheorem*{rmk}{Remark}
\newtheorem*{rmks}{Remarks}
\newtheorem*{lt}{Last time}
}
\newtheorem*{lem}{Lemma}
\usepackage{color}
\usepackage{CJK}
\usepackage{mathrsfs}
\title{Chapter I: The Fundamental Group}
\author{Xiyu Zhai}
\date{}							% Activate to display a given date or no date

\newcommand{\spec}[0]{\text{Spec}}
\newcommand{\oo}[0]{\mathscr{O}}
\newcommand{\rt}[0]{\rightarrow}
\begin{document}
\maketitle
\tableofcontents
\section{Basic Constructions}
\begin{lem}[1.15]
	If a space $X$ is the union of a collection of path-connected open sets $A_\alpha$ each containing the basepoint $x_0\in X$ and if each intersection $A_\alpha\cap A_\beta$ is path-connected, then every loop in $X$ at $x_0$ is homotopic to a product of loops each of which is contained in a single $A_\alpha$.
\end{lem}
\begin{ex}
	Obvious
\end{ex}
\begin{ex}
	Obvious
\end{ex}
\begin{ex}
	Obvious
\end{ex}
\begin{ex}
	Obvious
\end{ex}
\begin{ex}
	Obvious
\end{ex}
\begin{ex}
	Obvious
\end{ex}
\begin{ex}
	Obvious
\end{ex}
\begin{ex}
	Consider $S^1\times S^1\twoheadrightarrow S^1\hookrightarrow \mathbb{R}^2$, we have
	\begin{equation}
		f(x,y)\neq f(-x,-y)
	\end{equation}.
\end{ex}
\begin{ex}
	Define $S^2\to \mathbb{R}^2$ as follows:

	for each $\bm{n}\in S^2$, choose an average $t(\bm{n})\in \mathbb{R}$ such that plane $p_{\bm{n}}:\bm{n}\cdot \bm{x}=t$ cut $A_3$ in equal half, by "average" we mean the average of the maximal possible value and the minimal possible value so that $t(\bm{n})=-t(-\bm{n})$, i.e. $p_{\bm{n}}=p_{-\bm{n}}$. Let $x(\bm{n}),y(\bm{n})$ be the area of $A_1,A_2$ intersecting with the upper space divided by $p_{\bm{n}}$ along $\bm{n}$. We then have $x(\bm{n})+x(-\bm{n})=1$ and $y(\bm{n})+y(-\bm{n})=1$. Then apply Borsuk-Ulam theorem to $\bm{n}\mapsto (x(\bm{n}),y(\bm{n}))$ we are done.
\end{ex}
\begin{ex}
	Obvious
\end{ex}
\begin{ex}
	Obvious
\end{ex}
\begin{ex}
	Obvious
\end{ex}
\begin{ex}
	Obvious
\end{ex}
\begin{ex}
	Obvious
\end{ex}
\begin{ex}
	Obvious
\end{ex}
\begin{ex}
	Obvious
\end{ex}
\begin{ex}
	Obvious
\end{ex}
\begin{ex}
	Straightforward?
\end{ex}
\end{document}